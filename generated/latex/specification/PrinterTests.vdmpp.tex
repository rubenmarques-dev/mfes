\begin{vdmpp}[breaklines=true]
class PrinterTests is subclass of Tests
types
-- TODO Define types here
values
-- TODO Define values here
instance variables
company : Company := new Company();
client : Client := new Client("Ruben","povoa", company);
printer : Printer := new Printer("FEUP", client);
printer2 : Printer := new Printer("FEP", client);
user1 : User := new User("utilizador1", "password", client);
user2 : User := new User("utilizador2", "password", client);
document1 : [Document] := nil;
employee : [Employee] := nil;
operations
(*@
\label{testFunctionalPrinter:16}
@*)

private testFunctionalPrinter : () ==> ()
testFunctionalPrinter() ==
(
 assert(printer.checkFunctionalPrinter() = true);
);
(*@
\label{testHasBlack:22}
@*)

private testHasBlack : () ==> ()
testHasBlack() == 
(
 assert(printer.checkValidPrintBlack(200) = false);
);
(*@
\label{testHasColor:28}
@*)

private testHasColor : () ==> ()
testHasColor() == 
(
 assert(printer.checkValidPrintColor(100) = true);
);
(*@
\label{testHaveNoUserLogged:34}
@*)

private testHaveNoUserLogged : () ==> ()
testHaveNoUserLogged() == 
(
assert(printer.isFree() = true);
);
(*@
\label{testLogin:40}
@*)

private testLogin : () ==> ()
testLogin() == 
(
printer.login(user1);
assert(printer.getUserLogged() = user1);
printer.login(user2);
assert(printer.getUserLogged() = user1);
);
(*@
\label{testLogout:49}
@*)

private testLogout : () ==> ()
testLogout() == 
(
printer.login(user1);
assert(printer.getUserLogged() = user1);
printer.logout();
assert(printer.getUserLogged() = nil);
);
(*@
\label{testPossiblePrintBlackDocument:58}
@*)

private testPossiblePrintBlackDocument: () ==> ()
testPossiblePrintBlackDocument() ==
(
document1 := user1.createBlackDocument(6, "title_pb");
assert(printer.possiblePrintBlackDocument(document1) = false);
printer.login(user1);
assert(printer.possiblePrintBlackDocument(document1) = false);
user1.addToBalance(25);
assert(printer.possiblePrintBlackDocument(document1) = false);
user1.addToBalance(5);
assert(printer.possiblePrintBlackDocument(document1) = true);
printer.functional := false;
assert(printer.possiblePrintBlackDocument(document1) = false);
);
(*@
\label{testPossiblePrintColorDocument:73}
@*)

private testPossiblePrintColorDocument: () ==> ()
testPossiblePrintColorDocument() ==
(
document1 := user1.createColorDocument(50, "title_cor");
assert(printer.possiblePrintColorDocument(document1) = false);
printer.login(user1);
assert(printer.possiblePrintColorDocument(document1) = false);
user1.addToBalance(300);
assert(printer.possiblePrintColorDocument(document1) = false);
user1.addToBalance(160);
assert(printer.possiblePrintColorDocument(document1) = true);
printer.functional := false;
assert(printer.possiblePrintColorDocument(document1) = false);
);
(*@
\label{testPrintBlackDocument:88}
@*)

private testPrintBlackDocument: () ==> ()
testPrintBlackDocument() ==
(
document1 := user1.createBlackDocument(6, "title_pb");
user1.addToBalance(30);
printer.printBlackDocument(document1);
assert(user1.getBalance() = 0);
assert(printer.black_remaining = 94);
assert(printer.sheets_remaining = 94);
);
(*@
\label{testPrintColorDocument:99}
@*)

private testPrintColorDocument: () ==> ()
testPrintColorDocument() ==
(
document1 := user1.createColorDocument(10, "title_pb");
user1.addToBalance(90);
printer.printColorDocument(document1);
assert(user1.getBalance() = 0);
assert(printer.black_remaining = 90);
assert(printer.cyan_remaining = 90);
assert(printer.yellow_remaining = 90);
assert(printer.magenta_remaining = 90);
assert(printer.sheets_remaining = 90);
);
(*@
\label{testPrintDocument:113}
@*)

private testPrintDocument: () ==> ()
testPrintDocument() == (
user1.addToBalance(102);
document1 := user1.createBlackDocument(6, "title_pb");
printer.printDocument(document1);
assert(printer.black_remaining = 94);
assert(printer.sheets_remaining = 94);
assert(user1.getBalance() = 72);
document1 := user1.createColorDocument(8, "title_cor");
printer.printDocument(document1);
assert(printer.black_remaining = 86);
assert(printer.sheets_remaining = 86);
assert(printer.cyan_remaining = 92);
assert(printer.yellow_remaining = 92);
assert(printer.magenta_remaining = 92);
assert(user1.getBalance() = 0);
);
(*@
\label{testCreateProblem:131}
@*)

private testCreateProblem: () ==> ()
testCreateProblem() == (
user1.addToBalance(455);
document1 := user1.createBlackDocument(91, "title_pb");
printer.printDocument(document1);
employee := printer.client.company.getEmployee("Joao");
assert(employee.getNumProblems() = 1);
);
(*@
\label{testEquals:140}
@*)

private testEquals: () ==> ()
(*@
\label{main:142}
@*)
testEquals() ==
(
assert(Printer`equals(printer,printer) = true);
assert(Printer`equals(printer,printer2) = false);
);

public static main: () ==> ()
 main() ==
 (
  new PrinterTests().testFunctionalPrinter();
  new PrinterTests().testHasBlack();
  new PrinterTests().testHasColor();
  new PrinterTests().testHaveNoUserLogged();
  new PrinterTests().testLogin();
  new PrinterTests().testLogout();
  new PrinterTests().testPossiblePrintBlackDocument();
  new PrinterTests().testPossiblePrintColorDocument();
  new PrinterTests().testPrintBlackDocument();
  new PrinterTests().testPrintColorDocument();
  new PrinterTests().testPrintDocument();
  new PrinterTests().testCreateProblem();
  new PrinterTests().testEquals();
 );
functions
-- TODO Define functiones here
traces
-- TODO Define Combinatorial Test Traces here
end PrinterTests
\end{vdmpp}
\bigskip
\begin{longtable}{|l|r|r|r|}
\hline
Function or operation & Line & Coverage & Calls \\
\hline
\hline
\hyperref[main:142]{main} & 142&100.0\% & 6 \\
\hline
\hyperref[testCreateProblem:131]{testCreateProblem} & 131&100.0\% & 3 \\
\hline
\hyperref[testEquals:140]{testEquals} & 140&100.0\% & 3 \\
\hline
\hyperref[testFunctionalPrinter:16]{testFunctionalPrinter} & 16&100.0\% & 3 \\
\hline
\hyperref[testHasBlack:22]{testHasBlack} & 22&100.0\% & 3 \\
\hline
\hyperref[testHasColor:28]{testHasColor} & 28&100.0\% & 9 \\
\hline
\hyperref[testHaveNoUserLogged:34]{testHaveNoUserLogged} & 34&100.0\% & 3 \\
\hline
\hyperref[testLogin:40]{testLogin} & 40&100.0\% & 3 \\
\hline
\hyperref[testLogout:49]{testLogout} & 49&100.0\% & 3 \\
\hline
\hyperref[testPossiblePrintBlackDocument:58]{testPossiblePrintBlackDocument} & 58&100.0\% & 3 \\
\hline
\hyperref[testPossiblePrintColorDocument:73]{testPossiblePrintColorDocument} & 73&100.0\% & 3 \\
\hline
\hyperref[testPrintBlackDocument:88]{testPrintBlackDocument} & 88&100.0\% & 3 \\
\hline
\hyperref[testPrintColorDocument:99]{testPrintColorDocument} & 99&100.0\% & 3 \\
\hline
\hyperref[testPrintDocument:113]{testPrintDocument} & 113&100.0\% & 3 \\
\hline
\hline
PrinterTests.vdmpp & & 100.0\% & 51 \\
\hline
\end{longtable}

